\english
\chapter{Applications and Features of Satellite Communications}

\section{Introduction}

Satellite communications provide unique solutions for a number of communications and strategic applications that are possible only using the satellite environment. These applications are typically `nice' applications, such as:
\begin{itemize}
\item	Multimedia communications provided to wide geographical areas with low population density; in this case the deployment of the satellite is less expensive than the corresponding terrestrial network to provide the same service;
\item 	Maritime communications (e.g. Inmarsat) also associated to radio-navigation systems;
\item 	Television broadcasting: in this case, a single signal is broadcasted for a wide community of users, making more economically efficient the use of the satellite when compared to a wide network of local broadcasting stations to cover the geographical area;
\item Earth observation and monitoring systems offering a distinctive potential for strategic (military and commercial) information gathering.
\end{itemize}

All of the above applications require from medium to high data rate communication links, which need to be implemented with the resources typical of a spacecraft. Clearly the onboard available power, weight and antenna size are limited: this leads to the requirement of promoting state-of-the art research in the field of modulation and coding, as well as synchronization techniques able to operate al close as possible to the ultimate limits of performances, at very low signal to noise ratios and saving as much as possible the used bandwidth, which, besides, has to be shared among various coexisting satellites as well as terrestrial applications.

Hence state-of-the-art satellite communications for Earth Observation and Tele\-communications (which are among the most appealing applications requiring high data rate modems) are characterized by the need for both maximum information throughput and minimum power and bandwidth consumption, requirements evidently contradictory with an inter-relation theoretically stated by the Shannon Bound. Additionally the available bandwidth, data rate and signal to noise ratio specifications vary according to each specific mission/from one mission to another. An ideal all-purpose MODEM unit, flexible for any mission application and hence commercially appealing, should have the following features:
\begin{itemize}
\item Flexible configurability of data rate and bandwidth allocated;
\item High variability of signal to noise ratio of operation;
\item Performance always at 1 dB maximum from the Shannon Capacity Bound (fixed modulation and coding quality);
\item Efficient usage of power and hardware resources for processing.
\end{itemize}
Such a modem would allow having unique off-the-shelf solution, matching the needs of almost all the missions for Earth Observation and extra-high-speed telecommunications.

The novel DVB-S2 modulation and coding scheme as well as the Modem for Higher Order Modulation Schemes (MHOMS) performance are so close to the Shannon Bound that they are expected to set the modulation and coding baseline for many years to come. The MHOMS program was financed by European Space Agency (ESA). The envisaged MHOMS modem application scenarios will encompass as a minimum:
\begin{itemize}
\item High-speed distributed Internet access;
\item Trunk connectivity (backbone/backhaul);
\item Earth observation high-speed downlink;
\item Point-to-multipoint applications (e.g., high-speed multicasting/broadcasting).
\end{itemize}
Further details on MHOMS are available in \cite{b:mhoms,b:mhomsearthobs}.

\section{Remote Sensing and Earth Observation}

Earth observation satellites have a key role to play in both the civilian and military sectors, monitoring nearly all aspects of our planet. This is the only technology capable of providing truly global coverage, particularly over the vast expanses of the oceans and sparsely populated land areas (e.g. deserts, mountains, forests, and polar regions).
Earth observation data is being used by more than 300 research teams. Small high-tech firms, large corporations and public agencies (meteorological offices, etc.) use Earth-observation data for both operational and commercial purposes.

%Thales Alenia Space is World leader in geostationary weather satellites, with proven expertise in low-orbit Earth observation missions, European leader in Earth observation ground segments. Furthermore,
To provide professional services such as oceanography, meteorology, climatology, etc. to the end user, Earth observation satellites requires an high capacity of acquiring and processing a large volume of images on a daily basis. For example, COSMO-Sky-Med, in full constellation configuration, can acquire up to \(560 \unit{GB}\) per day, roughly corresponding to 1800 standard images.



COSMO-SkyMed interoperable system \cite{b:COSMOSKYMED} is a remarkable example of a telecommunications integrated system capable of supporting the ability to exchange data with other external observation systems according to agreed modalities and standard protocols. The COSMO-SkyMed (Constellation of four Small Satellites for Mediterranean basin Observation) is a low-orbit, dual-use Earth observation satellite system operating in the X-band. It is funded by the Italian Ministry of Research (MIUR) and Ministry of Defense and managed by the Italian space agency (ASI). COSMO-SkyMed 1 has been successfully launched on June 7th, 2007 and COSMO-SkyMed 2 on December 10, 2007.

Thales Alenia Space is the program prime contractor, responsible for the development and construction of the entire COSMO-SkyMed system, including the space segment, with four satellites equipped with high resolution X-band synthetic aperture radars (SAR) and the
ground segment, a sophisticated turnkey geographically distributed infrastructure. The ground segment is realized by Telespazio and includes the satellite control segment, mission planning segment, civilian user segment, defense user segment, mobile processing
stations and programming cells. Thales Alenia Space is also responsible for the mission simulator, the integrated logistics and all operations up to delivery of the constellation.

Three particular aspects characterize COSMO-SkyMed's system performance: the space constellation's short revisit time for global Earth coverage; the short system response time; and the multi-mode SAR instruments imaging performance.
COSMO-SkyMed features satellites using the most advanced remote sensing technology, with resolution of less than one meter, plus a complex and geographically distributed ground segment produced by Italian industry. COSMO-SkyMed's high image performance relies on the versatile SAR instrument, which can acquire images in different operative modes, and to generate image data with different (scene) size and resolution (from medium to very high) so as to cover a large span of applicative needs.

Synthetic Aperture Radars (SARs) are most employed to gather a large amount of high resolution images. In this field, Thales Alenia Space played an important role in the development of the X-SAR (SAR operating in X band) space and ground segments. X-SAR is a joint project between NASA, the German space agency (DARA), and the Italian space agency (ASI) as part of NASA's ``Mission to Planet Earth''.
X-SAR can measure virtually any region of the Earth in all weather and sunlight conditions, as well as penetrate vegetation, ice and dry sand, to map the surface and provide scientists with detailed information on climatic and geological processes, hydrological cycles and ocean circulation.
X-SAR has flown on the U.S. Space Shuttle twice, in April and September 1994, both times on Endeavour. The two consecutive flights observed the same sites on Earth during two different seasons in order to detect any seasonal changes, thus contributing to more precise and reliable monitoring of renewable and nonrenewable resources.
The SIR-C/X-SAR missions observed more than 400 sites. Based on the knowledge acquired during these missions, Thales Alenia Space participated in the Shuttle Radar Topography
Mission (January 2000) which re-used the X-SAR radars to produce the first three-dimensional map of the Earth's surface.

Concerning the applications, the space constellations and versatility of the multi-mode SAR instrument enable a wide variety of image data products with different sizes, resolutions, elaboration levels, and accuracies (e.g. geo-localization), suitable for different applications needs, and allowing large environmental monitoring, territory surveillance, intelligence applications, and critical phenomena detection. By interferometric images, three-dimensional Digital Elevation Model products can be obtained and used to render a city with buildings, streets, or high-resolution topography.

COSMO-SkyMed mission specifications are compatible with numerous applications relating to civilian and military/security domains with 2D or 3D data. 2D applications are mainly related to cartographic purpose. Other applications are aimed at:
\begin{itemize}
\item Agricolture, for crop mapping and management;
\item Landscape classification, for basic land use identification;
\item Geology mapping for mineral, oil, and gas resources detection;
\item Accurate urban area mapping for urban evolution surveys and management (identification of high-density and low-density urban zones), as well as industrial zone mapping with identification of buildings of interest, port implantations, storage zones, airports, etc.;
\item Road and network (electricity, gas, telecom) identification and mapping, for compiling the Geographic Information System (GIS) data base (note, e.g., its relevance for applications such as radio navigation and safety of life services);
\item Survey mapping.
\end{itemize}
Other applications fields concern hydrology, water resource inventory and detection, coastal zones for erosion and coastal line determination.
3D possible applications are various:
\begin{itemize}
\item \emph{Decision help}: based on 3D models as powerful tools. They allow citizens, city authorities and managements services to understand and explore a territory, and to build up a database for realistic simulations. For example, architecture, urbanization and new transport infrastructures can be simulated via a 3D database and presented to the different city players;
\item \emph{Natural and industrial risk management}: simulation, prevention and post-crisis. Using a Digital Elevation Model combined with a high-resolution image allows phenomena simulation such as flood, fire, earthquake, air pollution, etc. It can be used during different risk management phases, such as:
    \begin{itemize}
    \item Prevention: preparation of intervention missions, reports of risky areas to survey, simulation of phenomena and related interventions;
    \item Crisis anticipation: types of risky building, communication network failure, secondary itineraries to use, 3D localization of interesting positions (e.g. helicopter landing zones);
    \item Post-crisis: damage evaluation, reconstruction planning.
    \end{itemize}
\end{itemize}

In mid-2007 Italy and France have been activated a bilateral image-trading protocol providing for reciprocal access to Helios 2 and COSMO-SkyMed data. This agreement enabled the French and Italian armed forces to benefit from the complementary optical and radar image capabilities of these systems.

This complex, challenging system delivers outstanding security, international cooperation (for civilians and, within NATO countries, for defense), scalability and interoperability. In other words, COSMO-SkyMed provides Italy and partner countries with one of the world�s most technologically advanced observation systems to guarantee greater security and an improved standard of living.

\section{The Second Generation of Digital Video Broadcasting System}

DVB-S2 is the second-generation DVB specification for broadband satellite applications, developed on the success of the first generation specifications (i.e. DVB-S for broadcasting and DVB-DSNG for satellite news gathering and contribution services), benefiting from the technological achievements of the last decade.

The DVB-S2 system has been designed for various broadband satellite applications
\begin{itemize}
\item broadcasting services of Standard Definition TeleVision (SDTV) and High Definition TeleVision (HDTV);
\item interactive applications oriented to professional or domestic services (i.e. Interactive Services including Internet Access for consumer applications);
\item    professional service of Satellite News Gathering (SNG);
\item    distribution of TV signals (VHF/UHF) to Earth digital transmitters;
\item    data distribution and Internet Trunking.
\end{itemize}

The DVB-S2 standard achieves an higher capacity of data transmission than the first generation systems, an enhanced flexibility, a reasonable complexity
\footnote{More precisely, the complexity is referred to amount of total operations to perform, but it is a quantity difficult to define, since it is strongly dependent on the kind of implementation (analogic, digital) either implemented in hardware or software and on the selected design options.}
of the receiver. In fact, it has been specified around three key concepts: best transmission performance, total flexibility and reasonable receiver complexity.

To obtain a well balanced ratio between complexity and performance, DVB-S2 benefits from more recent developments in channel coding and modulation. Employment of these novel techniques translates in a 30 percent of capacity increase over DVB-S under the same transmission conditions.

To achieve the top quality performance, DVB-S2 is based on LDPC (Low Density Parity Check) codes, simple block codes with very limited algebraic structure, discovered by R. Gallager in 1962.
LDPC codes have an easily parallelizable decoding algorithm which consists of simple operations such as addition, comparison and table look-up; moreover the degree of parallelism is adjustable which makes it easy to trade-off throughput and complexity\footnote{For the sake of truth it has to be highlighted that, after the DVB-S2 contest for modulation and coding was closed, other coding schemes have been reconsidered by TAS-I for extra high speed operations, with less decoding complexity and almost equal performance. This has been made possible emulating the LDPC parallelism via ad-hoc designed parallel interleavers.}.
Their key characteristics, allowing quasi-error free operation at only 0,6 to 1,2 dB \cite{b:DVBstandard} from the Shannon limit, are:
\begin{itemize}
\item the very large LDPC code block length, 64 800 bits for the normal frame, and 16 200 bits for the short frame (remind that block size is in general related to performance);
\item the large number of decoding iterations (around 50 Soft-Input-Soft-Output (SISO) iterations);
\item the presence of a concatenated BCH outer code (without any interleaver), defined by the designers as a ``cheap insurance against unwanted error floors at high C/N ratios''.
\end{itemize}

Variable Coding and Modulation (VCM) functionalities allows different modulation and error protection levels which can be switched frame by frame. The adoption of a return channel conveying the link conditions allows to achieve closed-loop Adaptive Coding and Modulation (ACM).

High DVB-S2 flexibility allows to deal with any existing satellite transponder characteristics, with a large variety of spectrum efficiencies and associated C/N requirements. Furthermore, it is not limited to MPEG-2 video and audio source coding, but it is designed to handle a variety of audio-video and data formats including formats which the DVB Project is currently defining for future applications. Since video streaming applications are subjected to very strict time-constraints (around \(100 \unit{ms}\) for real-time applications), the major flexibility of present standard can be effectively exploited so as to employ \emph{unequal error protection} \cite{b:DVBstandardTR, b:UEP1, b:UEP2} techniques and differentiated service level (i.e. priority in the delivery queues). This kind of techniques are devoted to obtain, even in the worst transmission conditions, a graceful degradation of received video quality. More details about these interesting unicast IP services are in \cite{b:DVBstandardTR}.

Addressing into more details the possible applications and the relevant definitions, we observe that
the DVB-S2 system has been optimized for the following broadband satellite application scenarios.
\begin{description}
\item[Broadcast Services (BS):] digital multi-programme Television (TV)/High
Definition Television (HDTV) broadcasting services.
DVB-S2 is intended to provide Direct-To-Home (DTH) services for consumer Integrated Receiver Decoder (IRD), as well as collective antenna systems (Satellite Master Antenna Television - SMATV) and cable television head-end stations.
DVB-S2 may be considered a successor to the current DVB-S standard, and may be introduced for new services and allow for a long-term migration.
BS services are transported in MPEG Transport Stream format. VCM may be applied on multiple transport stream to achieve a differentiated error protection for different services (TV, HDTV, audio, multimedia).

\item[Interactive Services (IS):] interactive data services including internet
access.

DVB-S2 is intended to provide interactive services to consumer IRDs and to personal computers.
No recommendation is included in the DVB-S and DVB-S2 standards as far as the return path is concerned.
Therefore, interactivity can be established either via terrestrial connection through telephone lines, or via satellite.
Data services are transported in (single or multiple) Transport Stream format or in (single or multiple) generic stream format.
DVB-S2 can provide Constant Coding and Modulation (CCM), or ACM, where each individual satellite receiving station controls the protection mode of the traffic addressed to it.
\item[DTVC and DSNG:] Digital TV Contribution and Satellite News Gathering.
Digital television contribution applications by satellite consist of point-to-point or point-to-multipoint transmissions, connecting fixed or transportable uplink and receiving stations.
Services are transported in single (or multiple) MPEG Transport Stream format.
DVB-S2 can provide Constant Coding and Modulation (CCM), or Adaptive Coding and Modulation (ACM).
In this latter case, a single satellite receiving station typically
controls the protection mode of the full multiplex.
\end{description}



