\chapter{Conclusions} \label{ch:Conclusions}

In this thesis we have dealt with a digital communications TX section which can provide a reliable information transmission via satellite while complying with the challenging requirements on performance imposed by DVB-S2 standard.

My most working contributions to the overall project of the DVB-S2 TX section developed by TAS-I have been focused specifically on
\begin{itemize}
\item study of the theory underlying the BCH code: Galois fields and cyclic codes theory in order to have the theoretical basis to understand the DVB-S2 code structure and analyze algorithms related to;
\item theoretical analysis of DVB-S2 BCH code structure from which we have reach the conclusion that the DVB-S2 outer code (BCH) must be \emph{primitive} and \emph{shortened};
\item study of the BCH encoding procedures and algorithms and its related (serial) architectures;
\item modelling and design of a parallel BCH encoder with an high degree of parallelism in order to best match the TX Section (developed by TAS) speed requirements;
\item development of a Berlekamp-Massey software decoder in order give a proof of the error correction capacity of the code (encoder); to this purpose we have defined specifical routines aimed to compute arithmetic tables useful in decoding and error detection computations;
\item development of a software package (C/C++) in order to verify the functioning of architectures envisaged and give support to VLSI designers in validation of the correspondent VHDL model;
\end{itemize}

Concerning to the DVB-S2 architecture and overall TX section and thanks to the contribution of the TAS-I Algorithm \& Architectures team, this thesis has dealt with:
\begin{itemize}
\item identification of the strategic importance of satellite communications in some application scenarios such as Multimedia, Earth observation, radio-localization and navigation, \emph{etc.};
\item review of novel techniques aimed at obtaining top performance by handling modulation and coding schemes jointly;
\item assessment of the features of ACM techniques, which allow to better utilization of resources onboard while ensuring an high QoS (Quality of Service);
\item design of the DVB-S2 Modem architecture and blocks focusing on concatenated coding (BCH-LDPC) and modulation schemes;
\item qualitative description of LDPC encoding/decoding algorithms, and on the latter we have highlighted the importance of the BCH code to counteract against error floors affecting LDPC \emph{iterative decodings};
\item BCH encoding procedures and algorithms and its related (serial) architectures;
\item modelling and design of a parallel encoder with an high degree of parallelism in order to best match the TX Section (developed by TAS-I) speed requirements;
\item selection between two possible (and functionally equivalent) BCH parallel architectures in order to work in ACM;
\item development of a Berlekamp-Massey software decoder in order give a proof of the error correction capacity of the code (encoder); to this purpose we have defined specifical routines aimed to compute arithmetic tables useful in decoding and error detection computations;
\item development of a software package (C/C++) in order to verify the functioning of architectures envisaged and give support to VLSI designers in validation of the correspondent VHDL model;
\item participation to a preliminary laboratory test and validation of the TAS-I developed DVB-S2 TX section  .
\end{itemize}
%Before the equipment production, the work in progress will be devoted to
%\begin{itemize}
%\item evaluate and thus analyze BER curves by simulations
%\item measurement and BER analysis by a commercial BER meter
%\item compare BERs obtained by the BER meter with those theoretically expected
%\end{itemize}

