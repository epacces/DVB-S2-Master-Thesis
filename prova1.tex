
%% Serial Architecture (Type 2: serial fetching)
\begin{signalflow}[node distance=10mm]%
   % The \tikzgrid environment creates a fixed sized grid, where each
   % node in the grid is placed using the familiar array syntax.
   % The grid size is set using the node distance option.
   % Tip: Use the xscale and yscale options to get different spacing in the
   %      x and y directions.
  \tikzgrid{
      % building blocks
      \node[coordinate]       (c1)  {}           &&
      \node[node]       (n1)  {}           &&
      \node[node]       (n2)  {}           & &
      \node[coordinate] (c2)  {}           &
      \node[coordinate] (c3)  {}           &
      \node[node] (n3)        {}           & &
      \node[node] (n4) {}                  &
      \node[coordinate] (c4)  {}
      \\
        & &
      \node[multiplier] (m1)  {$g_1$} & &
      \node[multiplier] (m2)  {$g_2$} & & & &
      \node[multiplier] (mn)  {$g_{r-1}$}
      \\
      \node[coordinate]  (a1)                 &
      \node[delay]      (b1) {$x_0$}       &
      \node[adder]      (a2)  {}           &
      \node[delay]      (b2) {$x_1$}       &
      \node[adder]      (a3)  {}           &
      \node[delay]      (b3) {$x_2$}       &
      \node[coordinate] (c5) {}            &
      \node[coordinate] (c6) {}            &
      \node[adder] (a4)  {}                &
     \node[delay]      (br) {$x_{r-1}$}   &
     \node[node]        (ny) {}     &
      \node[adder]      (cr)  {}           &
      \\ &&&&&&&&&
      \node[input] (in) {$u(i)$}       & &
      \node[node]  (n9) {}
      \\ &&&&&&&&&&
      \node[node] (n10) {}
    }
  % signal paths
  % r is short hand notation for a real signal.
  % Use c to get a complex style signal
  \path[r>]  (in)--(n9);
  \path[r] (c1)--(a1);
  \path[r>] (a1)--(b1);
  \path[r]  (br)--(cr);
  \path[r]  (cr)--(c4);
  \path[r>] (c4)--(n3);
  \path[r>] (n3)--(mn);
  \path[r>] (n3)--(c3);
  \path[r.] (c3)--(c2);
  \path[r>] (n2)--(n1);
  \path[r]  (n1)--(c1);
  \path[r>] (b3)--(c5);
  \path[r.] (c5)--(c6);
  \path[r>] (c6)--(a4);
  \path[r>] (a4)--(br);
  \path[r>] (mn)--(a4);
  \path[r>] (c2)--(n2);
  \path[r>] (n2)--(m2);
  \path[r>] (n1)--(m1);
  \path[r>] (m1)--(a2);
  \path[r>] (m2)--(a3);
  \path[r>] (b1)--(a2);
  \path[r>] (a2)--(b2);
  \path[r>] (b2)--(a3);
  \path[r>] (a3)--(b3);
 % \path[r>] (br)--(n7);
%  \path[r>] (n7)--(cr);
\end{signalflow}
